This style has already been discussed.

\@specialtrue
%\setstyle{53}
\cxset{appendix name/.code=\gdef\appendixname{#1}}

\cxset{steward,
  appendix name=Appendix,
  numbering=arabic,
  custom=tikzspecial,
  offsety=0cm,
  image=rainbow,
  texti={So far we have seen how to reset styles for the common sectioning commands, such as chapters and sections. Other common elements of a book such as an appendices are discussed here.},
  textii={We have already investigated some of the applications of derivatives, but now that we know the differentiation
rules we are in a better position to pursue the applications of differentiation in greater depth. Here
we learn how derivatives affect the shape of a graph of a function and, in particular, how they help us
locate maximum and minimum values of functions. Many practical problems require us to minimize a
cost or maximize an area or somehow find the best possible outcome of a situation. In particular, we will
be able to investigate the optimal shape of a can and to explain the location of rainbows in the skys.}
}

\appendix
\cxset{numbering=Alpha,
          section numbering=numeric,
          section indent=0pt,
          section beforeskip=10pt,
          section afterskip=10pt}
\chapter{STYLING APPENDICES}

\section{Appendix section}

As far as LaTeX is concerned, there is nothing special in styling an appendix. It is either a chapter or a section with a different name. This name in order to allow internationalization is called \lstinline{appendixname}.
\bigskip

\begin{tcolorbox}[width=\linewidth]
\begin{lstlisting}
\newcommand\appendix{\par
  \setcounter{chapter}{0}%
  \setcounter{section}{0}%
  \gdef\@chapapp{\appendixname}(*@\footnote{The actual literal used for   \textbackslash{appendixname} is defined later on, so that you can customize the language}\label{appendixname}@*)
  \gdef\thechapter{\@Alph\c@chapter}
}
\end{lstlisting}
\end{tcolorbox}
\medskip

The code above is only a simplified version of the command. One might need to add more formatting information such as resetting equation numbers, tables and figures and any special floating environments that have their own numbering.

\begin{tcolorbox}[width=\linewidth]
\begin{lstlisting}
\renewcommand\appendix{\par
                \stepcounter{chapter}
                \setcounter{chapter}{0}
                \stepcounter{section}
                \setcounter{section}{0}
                \setcounter{equation}{0}
                \setcounter{figure}{0}
                \setcounter{table}{0}
                \setcounter{footnote}{0}
  \def\@chapapp{\appendixname}%
  \renewcommand\thechapter{\@Alph\c@chapter}}
\end{lstlisting}
\end{tcolorbox}


\section{Usage}

With the \lstinline{classx} package appendices are formatted as chapters.

\begin{tcolorbox}[width=\linewidth]
\begin{lstlisting}
\appendix
\cxset{numbering=Alpha}
\chapter{STYLING APPENDICES}
\end{lstlisting}
\end{tcolorbox}

\subsection{Enhancements}
More enhancements are possible. For one we can get rid of the chapter, which semantically is not very good, one should have followed a similar style to that of the sections and the \lstinline!\appendix{Title}!. I am not sure if this wouldn't be a bit confusing to people.

\subsubsection{Appendices at end of chapters}
Some styles require appendices to be set at the end of each chapter. These type of appendices can also be added. However an appendix counter might need to be defined.

\paragraph{test paragraph}

\subparagraph{test subparagraph}


%\setstyle{13}
\cxset{numbering=Alpha, name=Appendix}
\@specialfalse%required to negate effect of special tikz picture.

\chapter{Another Appendix}
\section{Calling appendix styles}
\lipsum[1-3]


\@specialtrue
\cxset{steward,
  appendix name=Appendix,
  numbering=Alpha,
  custom=tikzspecial,
  offsety=0cm,
  image=rainbow,
  texti={So far we have seen how to reset styles for the common sectioning commands, such as chapters and sections. Other common elements of a book such as an appendices are discussed here.},
  textii={We have already investigated some of the applications of derivatives, but now that we know the differentiation
rules we are in a better position to pursue the applications of differentiation in greater depth. Here
we learn how derivatives affect the shape of a graph of a function and, in particular, how they help us
locate maximum and minimum values of functions. Many practical problems require us to minimize a
cost or maximize an area or somehow find the best possible outcome of a situation. In particular, we will
be able to investigate the optimal shape of a can and to explain the location of rainbows in the sky.}
}
