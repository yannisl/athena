
\@specialtrue
\cxset{steward,
  appendix name=Appendix,
  numbering=Alpha,
  custom=tikzspecial,
  offsety=0cm,
  image=doffer,
  texti={Backmatter material should follow the style set by the book chapters. Escaping from standard LaTeX formatting is as easy as making some simple changes to thebibliography environment.},
  textii={We have already investigated some of the applications of derivatives, but now that we know the differentiation
rules we are in a better position to pursue the applications of differentiation in greater depth. Here
we learn how derivatives affect the shape of a graph of a function and, in particular, how they help us
locate maximum and minimum values of functions. Many practical problems require us to minimize a
cost or maximize an area or somehow find the best possible outcome of a situation. In particular, we will
be able to investigate the optimal shape of a can and to explain the location of rainbows in the sky.}
}

\chapter{\bibname}
This can only be a small introduction to LaTeX's bibliography style modifications and we will only focus on the opening style.

\parindent1em

The standard \LaTeX\ environment for generating a list of references is called the \verb+thebibliography+ \cite{GOOSSENS94}. In its default implementation it automatically
generates an appropriate heading and implements a vertical list structure in which every publication is represented as a separate item.

\begin{tcolorbox}[title=LaTeX default thebibliography environment.]
\begin{lstlisting}
\newdimen\bibindent
\setlength\bibindent{1.5em}
\renewenvironment{thebibliography}[1]
     {%\chapter{\bibname}%
      \@mkboth{\MakeUppercase\bibname}{\MakeUppercase\bibname}%
      \list{\@biblabel{\@arabic\c@enumiv}}%
           {\settowidth\labelwidth{\@biblabel{#1}}%
            \leftmargin\labelwidth
            \advance\leftmargin\labelsep
            \@openbib@code
            \usecounter{enumiv}%
            \let\p@enumiv\@empty
            \renewcommand\theenumiv{\@arabic\c@enumiv}}%
      \sloppy
      \clubpenalty4000
      \@clubpenalty \clubpenalty
      \widowpenalty4000%
      \sfcode`\.\@m}
     {\def\@noitemerr
       {\@latex@warning{Empty `thebibliography' environment}}%
      \endlist}
\renewcommand\newblock{\hskip .11em\@plus.33em\@minus.07em}
\let\@openbib@code\@empty
\end{lstlisting}
\end{tcolorbox}

As you can observe the environment uses a list structure to produce the bibliography entries. As a matter of interest the environment uses a counter from the enumerate environment to save the use of an extra counter (\texttt{enumiv})\cite{source2e}.

Our only contribution here is to redefine the command slightly so that we can call a numbered or unnumbered chapter heading. Since we did all the appendices in a special chapter format, we kept the same style for all.\cite{companion}



%\newdimen\bibindent
%\setlength\bibindent{1.5em}
%\renewenvironment{thebibliography}[1]
%     {%\chapter{\bibname}%
%      \@mkboth{\MakeUppercase\bibname}{\MakeUppercase\bibname}%
%      \list{\@biblabel{\@arabic\c@enumiv}}%
%           {\settowidth\labelwidth{\@biblabel{#1}}%
%            \leftmargin\labelwidth
%            \advance\leftmargin\labelsep
%            \@openbib@code
%            \usecounter{enumiv}%
%            \let\p@enumiv\@empty
%            \renewcommand\theenumiv{\@arabic\c@enumiv}}%
%      \sloppy
%      \clubpenalty4000
%      \@clubpenalty \clubpenalty
%      \widowpenalty4000%
%      \sfcode`\.\@m}
%     {\def\@noitemerr
%       {\@latex@warning{Empty `thebibliography' environment}}%
%      \endlist}
%\renewcommand\newblock{\hskip .11em\@plus.33em\@minus.07em}
%\let\@openbib@code\@empty

%% some references used
%\begin{thebibliography}{GMS94}
%
%\bibitem[GMS94]{GOOSSENS94}
%Michel Goossens, Frank Mittelbach, and Alexander Samarin.
%\newblock \emph{The LaTeX Companion}.
%\newblock Addison-Wesley Publishing Company, 1994.
%
%\bibitem[Lam94]{LAMPORT94}
%Leslie Lamport.
%\newblock \emph{LaTeX: A Document Preparation System}.
%\newblock Addison-Wesley Publishing Company, second edition, 1994.
%
%\bibitem[McP88]{MCPHERSON88}
%Kent McPherson.
%\newblock `{Page Layout in LaTeX}'.
%\newblock \emph{TUGboat}, 9(1):78--82, April 1988.
%
%\bibitem[Wil02]{MEMOIR}
%Peter Wilson.
%\newblock \emph{The memoir class for configurable book typesetting}.
%\newblock November, 2002.
%\newblock (Available from CTAN in
%           \texttt{macros/latex/contrib/memoir})
%
%\bibitem[FM04]{source2e}
%Michel Goossens, Frank Mittelbach, and Alexander Samarin.
%\newblock \emph{The LaTeX source2e}.
%\newblock (Available from CTAN in
%            \texttt{macros/core/latex}.
%\end{thebibliography}

\cite[59]{companion}
\cite[see][]{companion}
\cite[see][59--63]{companion}

\section{Using Biblatex}

When you use biblatex rather than the default LaTeX styles--and you should--it is easier to get the formatting you want as the \cs{printbibliography} comes with its own set of keys.


\nocite{*}
\printbibliography[title=REFERENCES]
